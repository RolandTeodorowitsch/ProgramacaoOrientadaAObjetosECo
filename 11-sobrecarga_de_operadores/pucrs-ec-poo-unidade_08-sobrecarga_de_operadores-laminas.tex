\documentclass[aspectratio=169]{beamer}
\usepackage[utf8]{inputenc}
\usepackage[T1]{fontenc}
\usepackage[brazil]{babel}
\usepackage{ragged2e}
\usepackage{booktabs}
\usepackage{verbatim}
\usepackage{gensymb}
\usepackage{multirow}
\usepackage{xcolor,colortbl}
\definecolor{verde}{rgb}{0,0.5,0}
\usepackage{listings}
\lstset{
  language=C++,
  basicstyle=\ttfamily,
  keywordstyle=\color{blue},
  stringstyle=\color{verde},
  commentstyle=\color{red},
  extendedchars=true,
  showspaces=false,
  showstringspaces=false,
%  numbers=left,
%  numberstyle=\tiny,
  breaklines=true,
  backgroundcolor=\color{green!10},
  breakautoindent=true,
  captionpos=b,
  xleftmargin=0pt
}
\newcommand\setItemnumber[1]{\setcounter{enumi}{\numexpr#1-1\relax}}

\usetheme{AnnArbor}
\usecolortheme{orchid}
\usefonttheme[onlymath]{serif}

\AtBeginSection[]{
  \begin{frame}
  \vfill
  \centering
  \begin{beamercolorbox}[sep=8pt,center,shadow=true,rounded=true]{title}
    \usebeamerfont{title}\insertsectionhead\par%
  \end{beamercolorbox}
  \vfill
  \end{frame}
}

\title[\sc{Sobrecarga de Operadores}]{Sobrecarga de Operadores}
\author[Roland Teodorowitsch]{Roland Teodorowitsch}
%\institute[LP2 - EC - PUCRS]{Laboratório de Programação II - Curso de Engenharia de Computação - PUCRS}
\institute[POO - EC - PUCRS]{Programação Orientada a Objetos - ECo - Curso de Engenharia de Computação - PUCRS}
\date{11 de setembro de 2023}

\begin{document}
\justifying

%-------------------------------------------------------
\begin{frame}
	\titlepage
\end{frame}

%=======================================================
\section{Sobrecarga de Operadores}

%-------------------------------------------------------
\begin{frame}\frametitle{Sobrecarga de Operadores}
\begin{itemize}
	\item A sobrecarga de um operador deve ser feita criando-se uma função ou método cujo nome deve iniciar pela palavra \texttt{operator}
	\item Exemplos:
	\begin{itemize}
		\item Para sobrecarregar o comportamento do operador de soma o nome da função/método deve ser \texttt{operator+}
		\item Para sobrecarregar o operador ``\textgreater'' a função seria operator\textgreater
	\end{itemize}
	\item A sobrecarga de um operador para objetos de determinada classe pode ser feita \textbf{dentro} (com métodos) ou \textbf{fora} (com funções) da respectiva classe
\end{itemize}
\end{frame}

%-------------------------------------------------------
\begin{frame}\frametitle{Sobrecarga de Operadores com Funções}
\begin{itemize}
	\item Quando a sobrecarga for feita com uma função, ou seja, \textbf{fora de uma classe}, esta função deverá ter:
	\begin{itemize}
		\item 2 parâmetros caso o operador seja binário
		\item 1 parâmetro caso o operador seja unário
	\end{itemize}
\end{itemize}
\end{frame}

%-------------------------------------------------------
\begin{frame}[fragile]\frametitle{Sobrecarga de Operadores Binários com Funções}
\begin{itemize}
	\item No caso de um operador \textbf{binário}, como o de multiplicação (\texttt{*}), o primeiro parâmetro desta função representará o operando que fica à esquerda do operador e o segundo representará o operando que fica à direita
	\item No exemplo a seguir, o operador \texttt{+} é sobrecarregado através de uma função, para operar sobre dois objetos da classe \texttt{Ponto}
	\item O tipo do retorno da função deve corresponder ao tipo de dados a ser gerado pela operação. O tipo deste retorno pode ser inclusive \texttt{void}, caso o operador nada retorne
\begin{lstlisting}[language=C++,basicstyle=\ttfamily\small]
Ponto operator+ (Ponto &p1, Ponto &p2){
  Ponto temp;
  temp.setX(p1.getX()+p2.getX());
  temp.setY(p1.getY()+p2.getY());
  return temp;
}
\end{lstlisting}
\end{itemize}
\end{frame}

%-------------------------------------------------------
\begin{frame}[fragile]\frametitle{Sobrecarga de Operadores Unários com Funções}
\begin{itemize}
	\item No caso de um operador \textbf{unário}, como o de \texttt{!}, o primeiro parâmetro desta função representará o operando que fica à direita do operador
	\item No exemplo a seguir, o operador \texttt{!} é sobrecarregado através de uma função, para zerar os atributos de objetos da classe \texttt{Ponto}
	\item Neste caso o retorno é \texttt{void}, mas poderia ser de outro tipo caso fosse necessário
\begin{lstlisting}[language=C++,basicstyle=\ttfamily\small]
void operator! (Ponto &p1){
  p1.setX(0);
  p1.setY(0);
}
\end{lstlisting}
\end{itemize}
\end{frame}

%-------------------------------------------------------
\begin{frame}\frametitle{Sobrecarga de Operadores com Métodos}
\begin{itemize}
	\item Quando a sobrecarga for feita com um método de uma classe, ou seja, quando ela faz parte da definição da classe (é definida \textbf{internamente}), esta função deverá ter:
	\begin{itemize}
		\item 1 parâmetros caso o operador seja binário
		\item nenhum parâmetro caso o operador seja unário
	\end{itemize}
\end{itemize}
\end{frame}

%-------------------------------------------------------
\begin{frame}[fragile]\frametitle{Sobrecarga de Operadores Binários com Métodos}
\begin{itemize}
	\item No caso de um operador \textbf{binário}, como o de multiplicação (\texttt{*}), o parâmetro desta função representará o operando que fica à direita do operador. O operando que fica à esquerda é representado pelos atributos da classe
	\item No exemplo a seguir, o operador \texttt{+} é sobrecarregado através de um método, para operar sobre dois objetos da classe \texttt{Ponto}
	\item O tipo do retorno da função deve corresponder ao tipo de dados a ser gerado pela operação
\begin{lstlisting}[language=C++,basicstyle=\ttfamily\small]
Ponto Ponto::operator+(const Ponto &p) const {
  return Ponto(x+p.obtemX(), y+p.obtemY());
}
\end{lstlisting}
\end{itemize}
\end{frame}

%-------------------------------------------------------
\begin{frame}[fragile]\frametitle{Sobrecarga de Operadores Unários com Métodos}
\begin{itemize}
	\item No caso de um operador \textbf{unário}, como o de \texttt{!}, o método não terá parâmetros, sendo que os atributos da classe representarão o operando que fica à direita do operador
	\item No exemplo a seguir, o operador \texttt{!} é sobrecarregado através de um método para zerar os atributos de objetos da classe \texttt{Ponto}
	\item Neste caso o retorno é \texttt{void}, mas poderia ser de outro tipo caso fosse necessário
\begin{lstlisting}[language=C++,basicstyle=\ttfamily\small]
void Ponto::operator! (){
  this->x = 0;
  this->y = 0;
}
\end{lstlisting}
\end{itemize}
\end{frame}

%-------------------------------------------------------
\begin{frame}\frametitle{Sobrecarga com \texttt{cin} e \texttt{cout}}
\begin{itemize}
	\item A sobrecarga do operador \texttt{\textless{}\textless} permite definir como um objeto será mostrado em \texttt{cout}, que é um objeto da classe \texttt{ostream}
	\item A sobrecarga do operador \texttt{\textgreater{}\textgreater} permite especificar como um objeto será lido de \texttt{cin}, que é um objeto da classe \texttt{istream}
	\item No exemplo da página a seguir, os dois operadores são sobrecarregados para operarem com objetos da classe \texttt{Ponto}
\end{itemize}
\end{frame}

%-------------------------------------------------------
\begin{frame}[fragile]\frametitle{Sobrecarga com \texttt{cin} e \texttt{cout}}
\begin{lstlisting}[language=C++,basicstyle=\ttfamily]
ostream &operator<<(ostream &out,const Ponto &p) {
  out << p.str();
  return out;
}

istream &operator>>(istream &in,Ponto &p) {
  cout << "X: ";
  in >> p.x;
  cout << "Y: ";
  in >> p.y;
  return in;
}
\end{lstlisting}
\end{frame}

%-------------------------------------------------------
\begin{frame}\frametitle{Operadores que podem ser sobrecarregados}
\begin{itemize}
	\item É possível sobrecarregar os seguintes operadores:
	\begin{itemize}
		\item Unários:\\
		\texttt{+ ~ - ~ * ~ \& ~ \~{} ~ ! ~ ++ ~ -{}- ~ -\textgreater ~ -\textgreater{}*}
		\item Binários:\\
		\texttt{+ ~ - ~ * ~ / ~ \% ~ \^{} ~ \& ~ | ~ \textless{}\textless ~ \textgreater{}\textgreater{}\\
		+= ~ -= ~ *= ~ /= ~ \%= ~ \^{}= ~ \&= ~ |= ~ \textless{}\textless= ~ \textgreater{}\textgreater=\\
		\textless ~ \textless= ~ \textgreater ~ \textgreater= ~ == ~ != ~ \&\& ~ ||\\
		, ~ [] ~ ()\\
		new ~ new[] ~ delete ~ delete[]}
	\end{itemize}
	\item A  maioria  dos  operadores  em  C++ podem  ser  sobrecarregados,  exceto  os  seguintes operadores: \texttt{:: ~ . ~ .* ~ ?: ~ sizeof()}
	\item Os seguintes operadores \textbf{não} podem ser sobrecarregados como funções amigas (só como funções membro): \texttt{= ~ () ~ [] ~ -> ~ new ~ delete}
\end{itemize}
\end{frame}

%=======================================================
\section{Exemplo: Classe \texttt{Ponto}}

%-------------------------------------------------------
\begin{frame}[fragile]\frametitle{\texttt{Ponto.hpp}}
\fontsize{6pt}{6pt}\selectfont{
\lstinputlisting{src/Ponto/Ponto.hpp}
}
\end{frame}

%-------------------------------------------------------
\begin{frame}[fragile]\frametitle{\texttt{Ponto.cpp}}
\fontsize{6pt}{6pt}\selectfont{
\lstinputlisting[firstline=1,lastline=32]{src/Ponto/Ponto.cpp}
}
\end{frame}

%-------------------------------------------------------
\begin{frame}[fragile]\frametitle{\texttt{Ponto.cpp} (continuação)}
\fontsize{6pt}{6pt}\selectfont{
\lstinputlisting[firstline=33]{src/Ponto/Ponto.cpp}
}
\end{frame}

%-------------------------------------------------------
\begin{frame}[fragile]\frametitle{\texttt{PontoMain.cpp}}
\begin{columns}
\begin{column}{0.5\linewidth}
\fontsize{6pt}{6pt}\selectfont{
\lstinputlisting[firstline=1,lastline=32]{src/Ponto/PontoMain.cpp}
}
\end{column}
\begin{column}{0.5\linewidth}
\fontsize{6pt}{6pt}\selectfont{
\lstinputlisting[firstline=34	]{src/Ponto/PontoMain.cpp}
}
\end{column}
\end{columns}
\end{frame}

%=======================================================
\section{Exercício}

%-------------------------------------------------------
\begin{frame}\frametitle{Exercício 1}
\begin{enumerate}
	\setItemnumber{1}
	\item Definir uma classe que represente um círculo e que apresente as seguintes características:
	\begin{itemize}
		\item Métodos Privados para calcular: área do círculo ($\pi{}r^2$), circunferência do círculo ($2\pi{}r$)
		\item Construtores: \texttt{Circulo()} com centro em (0,0) e raio 1, \texttt{Circulo(double raio)} com centro em (0,0), \texttt{Circulo (double x, double y)} com raio 1, \texttt{Circulo (double x, double y, double raio)}
		\item Métodos Públicos para: obter posição X do centro do círculo, obter posição Y do centro do círculo, definir posição X do centro do círculo, definir posição Y do centro do círculo, definir o raio R do círculo, multiplicar o raio R do círculo por determinado fator (de escala), obter o raio do círculo, obter uma cadeia de caracteres com as informações básicas do círculo, obter a área do círculo, obter a circunferência do círculo
		\item Método para sobrecarregar o operador \texttt{\textless{}\textless} para imprimir os dados de um círculo
		\item Método para sobrecarregar o operador \texttt{\textgreater{}\textgreater{}} para ler os dados de um círculo do terminal
		\item Métodos para sobrecarregar os operadores \texttt{\textgreater{}} e \texttt{\textless{}} para comparar a área de dois círculos
	\end{itemize}
	Criar um programa principal para testar a classe.
\end{enumerate}
\end{frame}

%=======================================================
\section{Créditos}

%-------------------------------------------------------
\begin{frame}\frametitle{Créditos}
\begin{itemize}
	\item Estas lâminas contêm trechos de materiais disponibilizados pelos professores Rafael Garibotti e Márcio Pinho.
\end{itemize}
\end{frame}

%=======================================================
\section{Soluções}

%-------------------------------------------------------
\begin{frame}[fragile]\frametitle{\texttt{Circulo.hpp}}
\fontsize{6pt}{6pt}\selectfont{
\lstinputlisting{src/Circulo/Circulo.hpp}
}
\end{frame}

%-------------------------------------------------------
\begin{frame}[fragile]\frametitle{\texttt{Circulo.cpp}}
\fontsize{6pt}{6pt}\selectfont{
\lstinputlisting[firstline=1,lastline=32]{src/Circulo/Circulo.cpp}
}
\end{frame}

%-------------------------------------------------------
\begin{frame}[fragile]\frametitle{\texttt{CirculoMain.cpp}}
\fontsize{6pt}{6pt}\selectfont{
\lstinputlisting[firstline=1,lastline=32]{src/Circulo/CirculoMain.cpp}
}
\end{frame}

%-------------------------------------------------------
\end{document}

