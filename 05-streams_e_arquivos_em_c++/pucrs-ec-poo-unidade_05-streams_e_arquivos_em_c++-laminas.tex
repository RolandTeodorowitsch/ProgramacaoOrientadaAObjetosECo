\documentclass[aspectratio=169]{beamer}
\usepackage[utf8]{inputenc}
\usepackage[T1]{fontenc}
\usepackage[brazil]{babel}
\usepackage{ragged2e}
\usepackage{booktabs}
\usepackage{verbatim}
\usetheme{AnnArbor}
\usecolortheme{orchid}
\usefonttheme[onlymath]{serif}
\usepackage{gensymb}
\usepackage{multirow}
\usepackage{xcolor,colortbl}
\definecolor{verde}{rgb}{0,0.5,0}
\usepackage{listings}
\lstset{
  language=C++,
  basicstyle=\ttfamily,
  keywordstyle=\color{blue},
  stringstyle=\color{verde},
  commentstyle=\color{red},
  extendedchars=true,
  showspaces=false,
  showstringspaces=false,
%  numbers=left,
%  numberstyle=\tiny,
  breaklines=true,
  backgroundcolor=\color{green!10},
  breakautoindent=true,
  captionpos=b,
  xleftmargin=0pt
}
\newcommand\setItemnumber[1]{\setcounter{enumi}{\numexpr#1-1\relax}}

\usetheme{AnnArbor}
\usecolortheme{orchid}
\usefonttheme[onlymath]{serif}

\AtBeginSection[]{
  \begin{frame}
  \vfill
  \centering
  \begin{beamercolorbox}[sep=8pt,center,shadow=true,rounded=true]{title}
    \usebeamerfont{title}\insertsectionhead\par%
  \end{beamercolorbox}
  \vfill
  \end{frame}
}

\title[\sc{\emph{Streams} e Arquivos em C++}]{\emph{Streams} e Arquivos em C++}
\author[Roland Teodorowitsch]{Roland Teodorowitsch}
%\institute[LP2 - EC - PUCRS]{Laboratório de Programação II - Curso de Engenharia de Computação - PUCRS}
\institute[POO - EC - PUCRS]{Programação Orientada a Objetos - ECo - Curso de Engenharia de Computação - PUCRS}
\date{30 de agosto de 2023}

\begin{document}
\justifying

%-------------------------------------------------------
\begin{frame}
	\titlepage
\end{frame}

%=======================================================
\section{Streams}

%-------------------------------------------------------
\begin{frame}\frametitle{\emph{Streams} em C++}
\begin{itemize}
	\item C++ permite acesso a dispositivos de entrada e saída
	\item \emph{Streams} são abstrações que correspondem a fluxos de entrada e saída
	\item \texttt{cout} e \texttt{cin} são fluxos de entrada e saída para o terminal
	\begin{itemize}
		\item \texttt{cout} é usado para escrever
		\item \texttt{cin} é usado para ler
	\end{itemize}
	\item A mesma abstração poderia ser usada para compor \emph{strings}, como se estivéssemos escrevendo na tela
\end{itemize}
\end{frame}

%-------------------------------------------------------
\begin{frame}[fragile]\frametitle{\texttt{stringstream}}
\begin{columns}[T]
\begin{column}{0.5\linewidth}
\begin{itemize}
	\item Exemplo:
\lstinputlisting[basicstyle=\ttfamily\scriptsize]{src/stringstream1.cpp}
\end{itemize}
\end{column}
\begin{column}{0.5\linewidth}
\begin{itemize}
	\item Resultado:
\lstinputlisting[basicstyle=\ttfamily\scriptsize]{src/stringstream1.output}
\end{itemize}
\end{column}
\end{columns}
\end{frame}

%-------------------------------------------------------
\begin{frame}[fragile]\frametitle{Usando \texttt{stringstream} para criar método \texttt{str()} para objeto}
\lstinputlisting[basicstyle=\ttfamily\tiny]{src/stringstream2.cpp}
\end{frame}

%=======================================================
\section{Arquivos}

%-------------------------------------------------------
\begin{frame}\frametitle{\emph{Streams} em C++}
\begin{itemize}
	\item \emph{Streams}: arquivos texto, arquivos binários, \emph{sockets}, etc.
	\item Acesso a arquivos:
	\begin{itemize}
		\item Criação e gravação de dados em arquivos
		\item Leitura dos dados de arquivos
	\end{itemize}
	\item Exemplos:
	\begin{itemize}
		\item Gravação de um conjunto de números aleatórios em um \textbf{arquivo texto} em C++ (\texttt{escrita.cpp})
		\item Leitura desse arquivo (\texttt{leitura.cpp})
	\end{itemize}
\end{itemize}
\end{frame}

%-------------------------------------------------------
\begin{frame}[fragile]\frametitle{Escrevendo em um Arquivo Texto}
\begin{itemize}
	\item Crie um arquivo chamado \texttt{escrita.cpp} e acrescente a ele os trechos de código a seguir...
	\item O principal arquivo de cabeçalho para trabalhar com arquivos em C++ é \texttt{fstream}
	\item O início do cógigo-fonte será, portanto:
\lstinputlisting[firstline=1,lastline=10,basicstyle=\ttfamily\scriptsize]{src/escrita.cpp}
\end{itemize}
\end{frame}

%-------------------------------------------------------
\begin{frame}[fragile]\frametitle{Escrevendo em um Arquivo Texto}
\begin{itemize}
	\item O primeiro passo dentro de \texttt{main()} é criar uma instância de \texttt{ofstream}, que será o fluxo de saída associado ao arquivo:
\lstinputlisting[firstline=12,lastline=13,basicstyle=\ttfamily\scriptsize]{src/escrita.cpp}
	\item O próximo passo é abrir o arquivo para escrita:
\lstinputlisting[firstline=14,lastline=15,basicstyle=\ttfamily\scriptsize]{src/escrita.cpp}
Neste trecho de código:
	\begin{itemize}
		\item Usa-se o modo \texttt{ios::out}, que cria o arquivo e abre ele para escrita
		\item Cuidado: se esse modo for usado em arquivos já existentes, eles serão \textbf{apagados}!
		\item O método \texttt{is\_open()} retorna \texttt{false} se o arquivo não está aberto, então usa-se para testar se foi possível realizar a operação
	\end{itemize}
\end{itemize}
\end{frame}

%-------------------------------------------------------
\begin{frame}[fragile]\frametitle{Escrevendo em um Arquivo Texto}
\begin{itemize}
	\item Como em linguagem C, a função \texttt{srand()} é utilizada para inicializar o gerador de números aleatórios (semente)
\lstinputlisting[firstline=16,lastline=16,basicstyle=\ttfamily\scriptsize]{src/escrita.cpp}
	\item Início da gravação dos dados (cabeçalho do arquivo):
\lstinputlisting[firstline=17,lastline=18,basicstyle=\ttfamily\scriptsize]{src/escrita.cpp}
	\item Gravação dos TAM registros numéricos:
\lstinputlisting[firstline=19,lastline=23,basicstyle=\ttfamily\scriptsize]{src/escrita.cpp}
\end{itemize}
\end{frame}

%-------------------------------------------------------
\begin{frame}[fragile]\frametitle{Escrevendo em um Arquivo Texto}
\begin{itemize}
	\item Para finalizar, feche o arquivo:
\lstinputlisting[firstline=24,lastline=27	,basicstyle=\ttfamily\scriptsize]{src/escrita.cpp}
	\item Compile o programa \texttt{escrita.cpp}, depois execute-o e verifique se o arquivo \texttt{teste.txt} foi criado no diretório corrente
\end{itemize}
\end{frame}

%-------------------------------------------------------
\begin{frame}[fragile]\frametitle{Números Aleatórios em C++}
\begin{itemize}
	\item C++11 traz uma nova forma para gerar números randômicos
	\item Sintaxe:
\begin{lstlisting}
// Arquivo de cabecalho da biblioteca
#include <random>

// Obtem um numero randomico do HW:
std::random_device rd;
// Gerador de semente:
std::mt19937 eng(rd());
// Define a distribuicao, neste caso a partir de 1 ate 10:
std::uniform_int_distribution<> distr(1, 10);
// Gera o numero randomico:
int num = distr(eng);
\end{lstlisting}
\end{itemize}
\end{frame}

%-------------------------------------------------------
\begin{frame}[fragile]\frametitle{Lendo de um Arquivo Texto}
\begin{itemize}
	\item Crie um arquivo chamado \texttt{leitura.cpp} e acrescente a ele os trechos de código a seguir...
	\item Os arquivos de cabeçalho que devem ser incluídos são os mesmos do programa que faz a escrita:
\lstinputlisting[firstline=1,lastline=8,basicstyle=\ttfamily\scriptsize]{src/leitura.cpp}
\end{itemize}
\end{frame}

%-------------------------------------------------------
\begin{frame}[fragile]\frametitle{Lendo de um Arquivo Texto}
\begin{itemize}
	\item O primeiro passo dentro de \texttt{main()} é criar uma instância de \texttt{ifstream}, que será o fluxo de entrada associado do arquivo:
\lstinputlisting[firstline=10,lastline=11,basicstyle=\ttfamily\scriptsize]{src/leitura.cpp}
	\item O próximo passo é abrir o arquivo para leitura:
\lstinputlisting[firstline=13,lastline=14,basicstyle=\ttfamily\scriptsize]{src/leitura.cpp}
	Neste trecho de código, usa-se o modo \texttt{ios::in}, que abre o arquivo para leitura
\end{itemize}
\end{frame}

%-------------------------------------------------------
\begin{frame}[fragile]\frametitle{Lendo de um Arquivo Texto}
\begin{itemize}
	\item Ler o cabeçalho do arquivo:
\lstinputlisting[firstline=15,lastline=17,basicstyle=\ttfamily\scriptsize]{src/leitura.cpp}
\end{itemize}
\end{frame}

%-------------------------------------------------------
\begin{frame}[fragile]\frametitle{Lendo de um Arquivo Texto}
\begin{itemize}
	\item Ler cada um dos valores que estão gravados no arquivo, mas, em vez de ler linhas, os valores serão lidos como inteiros:
\lstinputlisting[firstline=18,lastline=22,basicstyle=\ttfamily\scriptsize]{src/leitura.cpp}
Sobre a leitura:
	\begin{itemize}
		\item Depois do cabeçalho, as próximas linhas são compostas de dois números: um contador e o valor armazenado
		\item Basta então fazer uma repetição, que terminará quando não houver mais dados no arquivo
		\item O método \texttt{good()} retorna \texttt{false} quando algo diferente acontecer (erro ou final do arquivo)
		\item O dado lido só é exibido na tela, se o método \texttt{fail()} retornar \texttt{false}
	\end{itemize}
\end{itemize}
\end{frame}

%-------------------------------------------------------
\begin{frame}[fragile]\frametitle{Lendo de um Arquivo Texto}
\begin{itemize}
	\item O laço pode ter terminado por dois motivos: ou houve um erro, ou o arquivo terminou
	\begin{itemize}
		\item No primeiro caso, o método \texttt{bad()} retorna \texttt{true}
		\item No segundo caso, o método \texttt{eof()} retorna \texttt{true}
	\end{itemize}
	\item Logo, uma situação de erro ocorre quando \texttt{bad()} retornou \texttt{true}, ou \texttt{eof()} retornou \texttt{false}:
\lstinputlisting[firstline=23,lastline=25,basicstyle=\ttfamily\scriptsize]{src/leitura.cpp}
\end{itemize}
\end{frame}

%-------------------------------------------------------
\begin{frame}[fragile]\frametitle{Lendo de um Arquivo Texto}
\begin{itemize}
	\item Para finalizar, feche o arquivo:
\lstinputlisting[firstline=26,lastline=29,basicstyle=\ttfamily\scriptsize]{src/leitura.cpp}
	\item Compile o programa \texttt{leitura.cpp}, depois execute-o e verifique se os dados do arquivo \texttt{teste.txt} estão sendo corretamente exibidos no terminal
	\item Mais informações em:
	\begin{itemize}
		\item \url{http://www.cplusplus.com/reference/istream/iostream/}
		\item \url{http://www.inf.pucrs.br/~pinho/PRGSWB/Streams/streams.html}
		\item \url{http://www.inf.pucrs.br/~flash/lapro2ec/aulas/streams/aula_streams.html}
	\end{itemize}
\end{itemize}
\end{frame}

%=======================================================
\section{Lista de Exercícios}

%-------------------------------------------------------
\begin{frame}\frametitle{Lista de Exercícios}
\begin{enumerate}
	\setItemnumber{1}
	\item Copie os códigos vistos em aula em dois arquivos (respectivamente, \texttt{escrita.cpp} para o programa que gera os números e os salva em arquivo e \texttt{leitura.cpp} para o programa que lê os números do arquivo, exibindo-os no terminal). Compile-os e teste-os para verificar a criação e leitura do arquivo ``teste.txt'' no diretório corrente.
\end{enumerate}
\end{frame}

%-------------------------------------------------------
\begin{frame}[fragile]\frametitle{Lista de Exercícios}
\begin{enumerate}
	\setItemnumber{2}
	\item Escreva um programa em C++ que permita controlar as presenças dos alunos de uma turma. O programa deve:
	\begin{enumerate}[a]
{\footnotesize
		\item Ler os dados da turma em um arquivo com o seguinte formato: linha inicial contendo o número de alunos e o número de presenças ou faltas. Na sequência aparece uma linha para cada aluno com: nome do aluno (considere nomes com apenas uma palavra) seguido pelas presenças já registradas desse aluno (``P'' para presença, ``F'' para falta). Crie uma classe \texttt{Aluno} (que possui nome e vetor parcialmente preenchido para as presenças) e trabalhe com um vetor parcialmente preenchido de objetos da classe \texttt{Aluno}. O conteúdo do arquivo poderia ser o seguinte:
\lstinputlisting[basicstyle=\ttfamily\tiny]{src/chamada.txt}
		\item Realizar uma chamada na data corrente, imprimindo o nome do aluno e perguntando se o aluno está presente (``P'') ou ausente (``F'').
		\item Imprimir um relatório indicando para cada aluno: número de presenças, número de faltas e porcentagem de aulas que o aluno assistiu.
		\item Gerar um novo arquivo com o controle de presenças atualizado.
}
	\end{enumerate}
\end{enumerate}
\end{frame}

%=======================================================
\section{Créditos}

%-------------------------------------------------------
\begin{frame}\frametitle{Créditos}
\begin{itemize}
	\item Estas lâminas contêm trechos de materiais disponibilizados pelos professores Rafael Garibotti e Matheus Trevisan.
\end{itemize}
\end{frame}

%=======================================================
\section{Soluções}

%-------------------------------------------------------
\begin{frame}[fragile]\frametitle{Exercício 1: \texttt{escrita.cpp}}
\lstinputlisting[basicstyle=\ttfamily\tiny]{src/escrita.cpp}
\end{frame}

%-------------------------------------------------------
\begin{frame}[fragile]\frametitle{Exercício 1: \texttt{leitura.cpp}}
\fontsize{6pt}{6pt}\selectfont{
\lstinputlisting{src/leitura.cpp}
}
\end{frame}

%-------------------------------------------------------
\begin{frame}[fragile]\frametitle{Exercício 2: \texttt{Aluno.hpp}}
\lstinputlisting[basicstyle=\ttfamily\tiny]{src/Aluno.hpp}
\end{frame}

%-------------------------------------------------------
\begin{frame}[fragile]\frametitle{Exercício 2: \texttt{Aluno.cpp} (primeira parte)}
\lstinputlisting[firstline=1,lastline=24,basicstyle=\ttfamily\tiny]{src/Aluno.cpp}
\end{frame}

%-------------------------------------------------------
\begin{frame}[fragile]\frametitle{Exercício 2: \texttt{Aluno.cpp} (segunda parte)}
\lstinputlisting[firstline=26,lastline=50,basicstyle=\ttfamily\tiny]{src/Aluno.cpp}
\end{frame}

%-------------------------------------------------------
\begin{frame}[fragile]\frametitle{Exercício 2: \texttt{chamada.cpp} (primeira parte)}
\fontsize{6pt}{6pt}\selectfont{
\lstinputlisting[firstline=1,lastline=28]{src/chamada.cpp}
}
\end{frame}

%-------------------------------------------------------
\begin{frame}[fragile]\frametitle{Exercício 2: \texttt{chamada.cpp} (segunda parte)}
\fontsize{6pt}{6pt}\selectfont{
\lstinputlisting[firstline=29,lastline=56]{src/chamada.cpp}
}
\end{frame}

%-------------------------------------------------------
\end{document}

