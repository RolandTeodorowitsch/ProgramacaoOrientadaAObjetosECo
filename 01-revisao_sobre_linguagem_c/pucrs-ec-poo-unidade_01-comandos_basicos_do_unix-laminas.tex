\documentclass[aspectratio=169]{beamer}
\usepackage[utf8]{inputenc}
\usepackage[T1]{fontenc}
\usepackage[brazil]{babel}
\usepackage{ragged2e}
\usepackage{booktabs}
\usepackage{menukeys}
\usepackage{verbatim}
\usepackage{listings}
\usepackage{tikz}
%\usetikzlibrary{shadows}

\newcommand*\keystroke[1]{%
  \tikz[baseline=(key.base)]
    \node[%
      draw,
      fill=white,
      drop shadow={shadow xshift=0.25ex,shadow yshift=-0.25ex,fill=black,opacity=0.75},
      rectangle,
      rounded corners=2pt,
      inner sep=1pt,
      line width=0.5pt,
      font=\scriptsize\sffamily
    ](key) {#1\strut}
  ;
}

\usetheme{AnnArbor}
\usecolortheme{orchid}
\usefonttheme[onlymath]{serif}

\title[\sc{Comandos Básicos do Unix}]{Comandos Básicos do Unix}
\author[Roland Teodorowitsch]{Roland Teodorowitsch}
%\institute[PSB - ES - PUCRS]{Programação de Software Básico - Curso de Engenharia de Software - PUCRS}
%\institute[FPROG - EP - PUCRS]{Fundamentos de Programação - Escola Politécnica - PUCRS}
\institute[POO - EC - PUCRS]{Programação Orientada a Objetos - ECo - Curso de Engenharia de Computação - PUCRS}
\date{7 de março de 2023}

\begin{document}
\justifying

%-------------------------------------------------------
\begin{frame}
	\titlepage
\end{frame}

%=======================================================
\section{Introdução}

%-------------------------------------------------------
\begin{frame}\frametitle{Frases}
\emph{``With a PC, I always felt limited by the software available.\\On Unix, I am limited only by my knowledge.''} (Peter J. Schoenster)\\
~\\
\emph{``UNIX is simple and coherent, but it takes a genius (or at any rate, a programmer)\\to understand and appreciate its simplicity.''} (Dennis Ritchie)\\
~\\
\emph{``Pretty much everything on the web uses those two things: C and UNIX.\\	The browsers are written in C.\\The UNIX kernel --- that pretty much the entire Internet runs on --- is written in C.''} (Rob Pike)\\
\end{frame}

%-------------------------------------------------------
\begin{frame}\frametitle{Principais características}
\begin{itemize}
	\item Multiusuário e multitarefa
	\item Grande variedade de ferramentas, incluindo interpretadores de comandos programáveis e interfaces gráficas
	\item Flexibilidade e simplicidade
	\item Orientado a arquivos e fluxos
	\item Eficiência
	\item Portabilidade
	\item Padronização
	\item Configurabilidade
\end{itemize}
\end{frame}

%-------------------------------------------------------
\begin{frame}\frametitle{Histórico}
\begin{itemize}
	\item Em 1969, Ken Thompson, na Bell Laboratories/AT\&T, escreve um SO em \emph{assembly} para o DEC PDP­7
	\item O nome Unix surgiu como um trocadilho para outro SO muito popular na época: o Multics (\emph{Multiplexed Information and Computing Service})
	\item Originalmente, o Unix foi chamado de ``Unics'' (\emph{UNiplexed Information and Computing System})
	\item Em 1973, Ken Thompson e Denis Ritchie, reescreveram o Unix em C (10.000 linhas, das quais 80 \% eram em C)
	\item Em 1975, o seu código foi foi cedido para universidades, quando começou a ser bastante utilizado
	\item Veja: \href{https://www.youtube.com/watch?v=cN00SbyoHiQ}{\emph{Unix History Project}}
	\item Outros destaques: System V, GNU, MINIX, Linux, etc.
\end{itemize}
\end{frame}

%=======================================================
\section{Sessão}

%-------------------------------------------------------
\begin{frame}\frametitle{Sessões}
\begin{itemize}
	\item Pode-se usar o Unix através de um gerenciador de janelas (comum, mas opcional) ou diretamente de um terminal (\emph{shell})
	\item Iniciando uma sessão
	\begin{itemize}
		\item Acesso via \emph{user name} (único) e senha
		\item O \emph{shell} indica que está apto a receber comandos através de um \emph{prompt} (\texttt{\$})
	\end{itemize}
	\item Encerrando uma sessão (no \emph{shell})
	\begin{itemize}
		\item O comando \texttt{exit} encerra uma sessão
		\item Também pode-se usar \keystroke{Ctrl}+\keystroke{D}
		\item \texttt{logout} pode ser usado para terminais onde houve ``\emph{login}''
	\end{itemize}
\end{itemize}
\end{frame}

%=======================================================
\section{Comandos}

%-------------------------------------------------------
\begin{frame}\frametitle{Formato dos comandos}
\begin{itemize}
	\item Comandos geralmente são mnemônicos (por exemplo, \texttt{ls} ou \texttt{cp}) e são seguidos de opções e argumentos
	\item Podem aparecer opções de dois tipos:
	\begin{itemize}
		\item menos e letra: \texttt{-a} (formato original)
		\item menos-menos e palavra: \texttt{-{}-all} (formato GNU)
	\end{itemize}
	\item Argumentos podem, por exemplo, ser arquivos e diretórios sobre os quais o comando será executado
	\item ``Filosofia'':
	\begin{itemize}
		\item Se um comando funciona sobre um arquivo e o arquivo não é especificado, então o usuário deve fornecer o conteúdo do arquivo pelo teclado
		\item Por exemplo:\\\texttt{cat arquivo.txt\\cat}
	\end{itemize}
\end{itemize}
\end{frame}

%-------------------------------------------------------
\begin{frame}\frametitle{Para saber mais...}
\begin{itemize}
	\item Há páginas de ajuda para comandos, chamadas de sistema, arquivos de configuração, etc.
	\item O comando \texttt{man} é usado para acessar as páginas de ajuda:\\
\begin{tabular}{lp{5cm}r}
\texttt{man man} & & \tiny{(mostra a página de ajuda do comando \texttt{man})}\\\hline
\texttt{man ls} & & \tiny{(mostra a página de ajuda do comando \texttt{help})}\\\hline
\texttt{man cat} & & \tiny{(mostra a página de ajuda do comando \texttt{cat})}\\\hline
\texttt{man sleep} & & \tiny{(mostram a página de ajuda do comando \texttt{sleep})}\\\hline
\texttt{man 1 sleep} & & \tiny{(idem)}\\\hline
\texttt{man 3 sleep} & & \tiny{(mostra a página de ajuda da chamada \texttt{sleep})}\\\hline
\texttt{man 5 passwd} & & \tiny{(mostra a página de ajuda para o arquivo \texttt{/etc/passwd})}\\\hline
\texttt{man fopen} & & \tiny{(mostra a página de ajuda para a chamada \texttt{fopen})}\\\hline
\texttt{man 3 fopen} & & \tiny{	(idem)}\\
\end{tabular}
	\item Alguns comandos aceitam também a opção \texttt{-{}-help}\\
\begin{tabular}{lp{3.4cm}r}
\texttt{ls -{}-help} & & \tiny{(faz com que o próprio comando \texttt{ls} mostre informações sobre suas opções)}\\
\end{tabular}
\end{itemize}
\end{frame}

%-------------------------------------------------------
\begin{frame}\frametitle{Comandos para data e hora}
\begin{tabular}{lp{1cm}r}
\texttt{cal} & & \tiny{(mostra o calendário do mês atual)}\\\hline
\texttt{cal 2022} & & \tiny{(mostra o calendário do ano 2022)}\\\hline
\texttt{cal 12 2022} & & \tiny{(mostra o calendário do mês 12 do ano 2022)}\\\hline
\texttt{ncal} & & \tiny{(mostra o calendário do mês atual em formato alternativo)}\\\hline
\texttt{ncal 2022} & & \tiny{(mostra o calendário do ano 2022 em formato alernativo)}\\\hline
\texttt{ncal 12 2022} & & \tiny{(mostra o calendário do mês 12 do ano 2022 em formato alternativo)}\\\hline
\texttt{date} & & \tiny{(mostra a data e hora corrente)}\\\hline
\texttt{date '+\%d/\%m/\%Y - \%H:\%I:\%M'} & & \tiny{(mostra a data e hora corrente no formato especificado)}\\
\end{tabular}
\end{frame}

%-------------------------------------------------------
\begin{frame}\frametitle{Verificando quem está logado}
\begin{tabular}{lp{2cm}r}	
\texttt{who} & & \tiny{(mostra usuários acessando o sistema)}\\\hline
\texttt{who am i} & & \tiny{(mostra acesso do usuário corrrente)}\\\hline
\texttt{finger} & & \tiny{(mostra informações sobre usuários acessando o sistema -- nem sempre está disponível)}\\\hline
\texttt{finger roland} & & \tiny{(mostra informações sobre um usuário específico -- nem sempre está disponível)}\\\hline
\texttt{last} & & \tiny{(mostra últimos acessos de usuários ao sistema)}\\
\end{tabular}
\end{frame}

%-------------------------------------------------------
\begin{frame}\frametitle{Manipulação de diretórios}
\begin{tabular}{lp{2cm}r}
\texttt{pwd} & & \tiny{mostra diretório corrente()}\\\hline
\texttt{cd} & & \tiny{(vai para o diretório base do usuário -- diretório \emph{home})}\\\hline
\texttt{cd \~{}} & & \tiny{(idem)}\\\hline
\texttt{cd Downloads} & & \tiny{(entra no diretório \texttt{Downloads})}\\\hline
\texttt{cd -} & & \tiny{(retorna para o diretório visitado anteriormente)}\\\hline
\texttt{cd ..} & & \tiny{(vai para o diretório superior -- diretório-pai)}\\\hline
\texttt{ls} & & \tiny{(mostra arquivos no diretório corrente)}\\\hline
\texttt{ls -l} & & \tiny{(mostra detalhes dos arquivos no diretório corrente)}\\\hline
\texttt{ls -l -a} & & \tiny{(mostra detalhes de todos os arquivos no diretório corrente -- incluindo arquivos ocultos)}\\\hline
\texttt{ls -la} & & \tiny{(idem)}\\\hline
\texttt{ls -l -{}-all} & & \tiny{(idem)}\\\hline
\texttt{ls /tmp} & & \tiny{(vai para o diretório \texttt{/tmp})}\\\hline
\texttt{mkdir \~{}/dir} & & \tiny{(cria o diretório \texttt{dir} no diretório base)}\\\hline
\texttt{rmdir \~{}/dir} & & \tiny{(remove o diretório \texttt{dir} -- que deve estar vazio -- no diretório base)}\\
\end{tabular}
\end{frame}

%-------------------------------------------------------
\begin{frame}\frametitle{Movimentação, cópia e remoção de arquivos e diretórios}
\begin{tabular}{lp{0.5cm}r}
\texttt{mv arqOrigem arqDestino} & & \tiny{(troca o nome do arquivo de \texttt{arqOrigem} para \texttt{arqDestino})}\\\hline
\texttt{mv -i arqOrigem arqDestino} & & \tiny{(idem, porém pedindo confirmação se for preciso sobrescrever \texttt{arqDestino})}\\\hline
\texttt{cp arqOrigem arqDestino} & & \tiny{(copia o arquivo \texttt{arqOrigem} sobre \texttt{arqDestino})}\\\hline
\texttt{cp -i arqOrigem arqDestino} & & \tiny{(idem, porém pedindo confirmação se for preciso sobrescrever \texttt{arqDestino})}\\\hline
\texttt{cp -r dir destino} & & \tiny{(copia o diretório \texttt{dir} e todos os seus conteúdos para o destino especificado)}\\\hline
\texttt{rm arq} & & \tiny{(remove o arquivo \texttt{arq}, sem confirmação)}\\\hline
\texttt{rm -i arq} & & \tiny{(remove o arquivo \texttt{arq}, pedindo uma confirmação)}\\\hline
\texttt{rm -r dir} & & \tiny{(remove o diretório \texttt{dir} e todos os seus conteúdos)}\\
\end{tabular}
\end{frame}

%-------------------------------------------------------
\begin{frame}\frametitle{Visualização de arquivos}
\begin{tabular}{lp{2cm}r}
\multicolumn{3}{l}{\texttt{cd \~{} ; for i in `seq 1 1 50` ; do echo linha \$i ;  done \textgreater{} arq.txt}}\\
& & \tiny{(cria o arquivo \texttt{arq.txt} com 50 linhas para teste)}\\\hline
\texttt{cat arq.txt} & & \tiny{(mostra o arquivo \texttt{arq.txt})}\\\hline
\texttt{more arq.txt} & & \tiny{(mostra o arquivo \texttt{arq.txt}, de forma paginada e controlada do início para o fim)}\\\hline
\texttt{less arq.txt} & & \tiny{(funciona como o comando \texttt{more}, podendo voltar em direção ao início do arquivos)}\\\hline
\texttt{head arq.txt} & & \tiny{(mostra o início de um arquivo, ou seja, suas 10 primeiras linhas)}\\\hline
\texttt{head -5 arq.txt} & & \tiny{(mostra as 5 primeiras linhas de um arquivo)}\\\hline
\texttt{tail arq.txt} & & \tiny{(mostra o final de um arquivo, ou seja, suas 10 últimas linhas)}\\\hline
\texttt{tail -5 arq.txt} & & \tiny{(mostra as 5 últimas linhas de um arquivo)}\\
\end{tabular}
\end{frame}

%-------------------------------------------------------
\begin{frame}\frametitle{Comandos diversos}
\begin{tabular}{lp{0.5cm}r}
\texttt{echo "Oi!"} & & \tiny{(mostra a mensagem ``Oi!'')}\\\hline
\texttt{echo \$PATH} & & \tiny{(mostra o conteúdo da variável de ambiente \texttt{PATH})}\\\hline
\texttt{tty} & & \tiny{(mostra o nome do terminal que está conectado à saída padrão)}\\\hline
\texttt{ls -l arq.txt} & & \tiny{(mostra informações completas sobre o arquivo \texttt{arq.txt})}\\\hline
\texttt{touch arq.txt} & & \tiny{(atualiza a data e hora de alteração do arquivo \texttt{arq.txt}}\\\hline
\texttt{ls -l arq.txt} & & \tiny{(mostra informações completas sobre o arquivo \texttt{arq.txt}, que agora tem nova data e hora)}\\\hline
\texttt{touch novo\_arq.txt} & & \tiny{(cria um arquivo \texttt{novo\_arq.txt}, sem conteúdo)}\\\hline
\end{tabular}
\end{frame}

%-------------------------------------------------------
\begin{frame}\frametitle{Permissões de acesso a arquivos (1/2)}
\begin{tabular}{lp{0.1cm}r}
\texttt{touch arq.txt} & & \tiny{(altera data/hora ou cria o arquivo \texttt{arq.txt})}\\\hline
\texttt{ls -l arq.txt} & & \tiny{(mostra as informações do arquivo \texttt{arq.txt})}\\\hline
\texttt{chmod a+x arq.txt} & & \tiny{(define permissões de execução para dono, grupo e outros no arquivo \texttt{arq.txt})}\\\hline
\texttt{ls -l arq.txt} & & \tiny{(mostra as informações do arquivo \texttt{arq.txt})}\\\hline
\texttt{chmod go-rwx arq.txt} & & \tiny{(remove permissões de leitura, escrita e execução de grupo e outros)}\\\hline
\texttt{ls -l arq.txt} & & \tiny{(mostra as informações do arquivo \texttt{arq.txt})}\\\hline
\texttt{chmod 751 arq.txt} & & \tiny{(define as permissões do arquivo para rwx r-x -{}-r)}\\\hline
\texttt{ls -l arq.txt} & & \tiny{(mostra as informações do arquivo \texttt{arq.txt})}\\\hline
\texttt{chown roland.users arq.txt} & & \tiny{(troca dono e grupo do arquivo -- só parar \texttt{root}/administrador)}\\
\end{tabular}
\end{frame}

%-------------------------------------------------------
\begin{frame}\frametitle{Permissões de acesso a arquivos (2/2)}
\begin{tabular}{lp{1.5	cm}r}
\texttt{rm arq.txt} & & \tiny{(remove o arquivo \texttt{arq.txt})}\\\hline
\texttt{umask 137} & & \tiny{(define a máscara de permissões para criação de novos arquivos como u=rwx,g=rx,o=rx)}\\\hline
\texttt{umask -S} & & \tiny{(mostra a máscara de permissões para criação de novos arquivos)}\\\hline
\texttt{ls -l arq.txt} & & \tiny{(mostra as informações do arquivo \texttt{arq.txt})}\\\hline
\texttt{rm arq.txt} & & \tiny{(remove o arquivo \texttt{arq.txt})}\\\hline
\texttt{umask 333} & & \tiny{(define a máscara de permissões para criação de novos arquivos como u=r,g=r,o=r)}\\\hline
\texttt{ls -l arq.txt} & & \tiny{(mostra as informações do arquivo \texttt{arq.txt})}\\\hline
\texttt{rm arq.txt} & & \tiny{(remove o arquivo \texttt{arq.txt})}\\\hline
\texttt{umask 337} & & \tiny{(define a máscara de permissões para criação de novos arquivos como u=r,g=r,o=)}\\\hline
\texttt{ls -l arq.txt} & & \tiny{(mostra as informações do arquivo \texttt{arq.txt})}\\\hline
\end{tabular}
\end{frame}

%-------------------------------------------------------
\begin{frame}\frametitle{\texttt{expr}}
\begin{itemize}
	\item É utilizado para fazer cálculos simples
	\item Deve-se usar sempre ``\texttt{\textbackslash}'' antes de símbolos (exceto ``\texttt{+}'' e ``\texttt{-}'')
	\item Exemplos:\\
\begin{tabular}{lp{2.5cm}r}
\texttt{expr 14 + \textbackslash( 10 \textbackslash* 4 \textbackslash)} & & \tiny{(retorna 54, ou seja, o resultado de $14 + 10 \times 4$)}\\\hline
\texttt{expr 5 \textbackslash$>$ 1} & & \tiny{(retorna 1, pois 5 é maior do que 1)}\\\hline
\texttt{expr 5 \textbackslash$<$ 1} & & \tiny{(retorna 0, pois 5 NÃO é menor do que 1)}\\
\end{tabular}
\end{itemize}
\end{frame}

%=======================================================
\section{Redirecionamento de E/S}

%-------------------------------------------------------
\begin{frame}\frametitle{Entrada}
\begin{itemize}
	\item No Unix, a entrada para um comando pode ser obtida de várias formas:\\
	\begin{itemize}
		\item Do próprio terminal:\\
\texttt{sort}
		\item De um arquivo aberto pelo próprio comando:\\
\texttt{sort /etc/passwd}
		\item De um arquivo aberto pelo \emph{shell} e passado para o comando por redirecionamento da entrada:\\
\texttt{sort $<$ /etc/passwd}
	\end{itemize}
	\item Portanto, em vez de esperar que o usuário digite o que deve ser processado, pode-se especificar o que deve ser processado a partir de um arquivo
\end{itemize}
\end{frame}

%-------------------------------------------------------
\begin{frame}\frametitle{Saída}
\begin{itemize}
	\item A saída dos programas pode ser redirecionada para um arquivo com ``\texttt{$>$}'':\\
\texttt{ls $>$ arquivo.txt}
	\item Neste caso, o arquivo será criado (ou recriado) sem nenhum conteúdo e a saída do comando será salva neste arquivo
	\item Para acrescentar no final do arquivo, sem recriá-lo, usa-se ``\texttt{$>>$}'':\\
\texttt{ls /etc $>>$ arquivo.txt}
	\item Existem 2 tipos de saídas
	\begin{itemize}
		\item Saída padrão: gerada por \texttt{printf(...)} ou \texttt{fprintf(stdout,...)}
		\item Saída de erro: gerada por \texttt{fprintf(stderr,...)}
	\end{itemize}
	\item Operadores de redirecionamento:
	\begin{itemize}
		\item ``\texttt{$>$}'' e ``\texttt{$>>$}'': redirecionam apenas a saída padrão
		\item ``\texttt{\&$>$}'': redireciona as duas saídas
		\item ``\texttt{1$>$}'': redireciona apenas a saída padrão
		\item ``\texttt{2$>$}'': redireciona apenas a saída de erro
	\end{itemize}
\end{itemize}
\end{frame}

%-------------------------------------------------------
\begin{frame}[fragile]\frametitle{Testes (1/2)}
\begin{itemize}
	\item \texttt{gedit aplic.c}
\begin{block}{}
\begin{lstlisting}[language=C,basicstyle=\ttfamily,keywordstyle=\color{red}]
#include <stdio.h>

int main() {
   fprintf(stdout,"informacao\n");
   fprintf(stderr,"erro\n");
   return 0;
}
\end{lstlisting}
\end{block}
	\item \texttt{cc aplic.c -o aplic}
	\item \texttt{./aplic}
\end{itemize}
\end{frame}

%-------------------------------------------------------
\begin{frame}\frametitle{Testes (2/2)}
\begin{itemize}
	\item \texttt{./aplic $>$ saida.txt}
	\item \texttt{cat saida.txt}
	\item \texttt{./aplic $>>$ saida.txt}
	\item \texttt{cat saida.txt}
	\item \texttt{rm saida.txt}
	\item \texttt{./aplic $>$ /dev/null}
	\item \texttt{./aplic \&$>$ /dev/null}
	\item \texttt{./aplic 1$>$ /dev/null}
	\item \texttt{./aplic 2$>$ /dev/null}
\end{itemize}
\end{frame}

%-------------------------------------------------------
\begin{frame}\frametitle{Questões}
\begin{enumerate}
	\item Qual a diferença entre os dois comandos a seguir?\\
\texttt{expr 5 $>$ 1\\
expr 5 \textbackslash$>$ 1}
	\item Qual a diferença entre a execução dos seguintes comandos?\\
\texttt{cat arq.txt | sort\\
sort arq.txt\\
sort $<$ arq.txt}
\end{enumerate}
\end{frame}

\end{document}
