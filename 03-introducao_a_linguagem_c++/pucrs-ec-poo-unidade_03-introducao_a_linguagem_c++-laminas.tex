\documentclass[aspectratio=169]{beamer}
\usepackage[utf8]{inputenc}
\usepackage[T1]{fontenc}
\usepackage[brazil]{babel}
\usepackage{ragged2e}
\usepackage{booktabs}
\usepackage{verbatim}
\usepackage{gensymb}
\usepackage{multirow}
\usepackage{xcolor,colortbl}
\definecolor{verde}{rgb}{0,0.5,0}
\usepackage{listings}
\lstset{
  language=C++,
  basicstyle=\ttfamily\small,
  keywordstyle=\color{blue},
  stringstyle=\color{verde},
  commentstyle=\color{red},
  extendedchars=true,
  showspaces=false,
  showstringspaces=false,
%  numbers=left,
%  numberstyle=\tiny,
  breaklines=true,
  backgroundcolor=\color{green!10},
  breakautoindent=true,
  captionpos=b,
  xleftmargin=0pt
}
\newcommand\setItemnumber[1]{\setcounter{enumi}{\numexpr#1-1\relax}}

\usetheme{AnnArbor}
\usecolortheme{orchid}
\usefonttheme[onlymath]{serif}

\AtBeginSection[]{
  \begin{frame}
  \vfill
  \centering
  \begin{beamercolorbox}[sep=8pt,center,shadow=true,rounded=true]{title}
    \usebeamerfont{title}\insertsectionhead\par%
  \end{beamercolorbox}
  \vfill
  \end{frame}
}

\title[\sc{Introdução à Linguagem C++}]{Introdução à Linguagem C++}
\author[Roland Teodorowitsch]{Roland Teodorowitsch}
%\institute[LP2 - EC - PUCRS]{Laboratório de Programação II - Curso de Engenharia de Computação - PUCRS}
\institute[POO - EC - PUCRS]{Programação Orientada a Objetos - ECo - Curso de Engenharia de Computação - PUCRS}
\date{14 de agosto de 2023}

\begin{document}
\justifying

%-------------------------------------------------------
\begin{frame}
	\titlepage
\end{frame}

%=======================================================
\section{Visão Geral}

%-------------------------------------------------------
\begin{frame}\frametitle{Extensão e Compilação}
\begin{itemize}
	\item Arquivos escritos em linguagem C++ são salvos com a extensão .cpp
	\item O compilador para C++ é o \texttt{g++} (funciona como o \texttt{gcc})
	\item Uso:\\\texttt{g++ -std=c++11 meu\_programa.cpp -o meu\_programa}
	\item Se não tiver sido criado nenhum \texttt{Makefile} (ainda), pode-se usar:\\\texttt{make meu\_programa}
\end{itemize}
\end{frame}

%-------------------------------------------------------
\begin{frame}[fragile]\frametitle{Estrutura Básica de um Programa}
\begin{columns}[T]
\begin{column}{0.5\linewidth}
\begin{enumerate}
	\item Inclusão de arquivos
	\item Definição do espaço de nomes (opcional)
	\item Declaração de macros, tipos e variáveis globais (quando houver)
	\item Métodos, incluindo o método \texttt{main()}
\end{enumerate}
\end{column}
\begin{column}{0.5\linewidth}
\begin{lstlisting}
#include <iostream>
#include <cmath>
#include <cstdlib>
#include <string>

using namespace std;

int main(){
  // Declaracao de variaveis
  // Comandos da funcao main
  return 0;
}
\end{lstlisting}
\end{column}
\end{columns}
\end{frame}

%-------------------------------------------------------
\begin{frame}\frametitle{Inclusões}
\begin{itemize}
	\item Arquivos de cabeçalho das bibliotecas não possuem a extensão \texttt{.h}
	\item Principais arquivos de cabeçalho:
	\begin{itemize}
		\item \texttt{iostream}: funções de entrada e saída
		\item \texttt{iomanip}: formatação de entrada e saída
		\item \texttt{cmath}: funções matemáticas
		\item \texttt{cstdlib}: funções de uso geral
		\item \texttt{ctime}: funções para manipulação de tempo e data
		\item \texttt{string}: funções para uso de \emph{strings}
	\end{itemize}
\end{itemize}
\end{frame}

%-------------------------------------------------------
\begin{frame}[fragile]\frametitle{Espaço de Nomes do Compilador}
\begin{itemize}
	\item Permite a definição de estruturas, classes, funções, etc.
	\item Permite duplicidade de identificadores, desde que eles estejam em espaços de nomes diferentes
	\item Por padrão, a linguagem utiliza o \emph{namespace} \texttt{std}
	\begin{itemize}
		\item Por exemplo, em \texttt{std::cout}, \texttt{cout} pertence ao \emph{namespace} \texttt{std}
	\end{itemize}
	\item Pode-se simplificar isto usando:
\begin{lstlisting}
using std::cout;
// ou
using namespace std;
\end{lstlisting}
\end{itemize}
\end{frame}

%-------------------------------------------------------
\begin{frame}[fragile]\frametitle{Espaço de Nomes -- Exemplo}
\lstinputlisting[basicstyle=\ttfamily\tiny]{src/namespaces.cpp}
\end{frame}

%-------------------------------------------------------
\begin{frame}[fragile]\frametitle{Função principal}
\begin{itemize}
	\item É a mesma usada em C
	\item É formada por declarações de variáveis (não precisam estar no início dos blocos) e comandos
\end{itemize}
\begin{lstlisting}
int main()
// ou recebendo parametros da linha de comando:
int main(int argc, char *argv[])
\end{lstlisting}
\end{frame}

%=======================================================
\section{Novos Tipos}

%-------------------------------------------------------
\begin{frame}[fragile]\frametitle{Novos Tipos}
\begin{itemize}
	\item C++ trouxe dois novos tipos
	\begin{itemize}
		\item \texttt{bool}
		\begin{itemize}
			\item Serve para valores lógicos
			\item Pode receber \texttt{true} ou \texttt{false}
		\end{itemize}
		\item \texttt{string}
		\begin{itemize}
			\item É uma forma mais prática de armazenar e trabalhar com cadeias de caracteres
			\item É uma classe e está definida no arquivo de cabeçalho \texttt{string}
			\item Não é preciso definir o tamanho
			\item Pode ser usado com operadores como, por exemplo, \texttt{==} ou \texttt{=} (evitando ter que usar métodos como \texttt{strcmp()} ou \texttt{strcpy()})
			\item Possui uma série de métodos definidos e prontos para serem usados (veja \url{https://cplusplus.com/reference/string/string/})
		\end{itemize}
	\end{itemize}
\end{itemize}
\end{frame}

%-------------------------------------------------------
\begin{frame}[fragile]\frametitle{\texttt{bool} e \texttt{string}}
\lstinputlisting[basicstyle=\ttfamily\scriptsize]{src/bool_e_string.cpp}
\end{frame}

%=======================================================
\section{Entrada e Saída}

%-------------------------------------------------------
\begin{frame}[fragile]\frametitle{Saída}
\begin{itemize}
	\item \texttt{std::cout} ou \texttt{cout} é o fluxo de saída padrão
	\item O operador \texttt{\textless{}\textless} (de inserção) é usado para enviar informações para a saída
\end{itemize}
\lstinputlisting{src/ola_turma.cpp}
\end{frame}

%-------------------------------------------------------
\begin{frame}[fragile]\frametitle{Saída}
\begin{itemize}
	\item A quebra de linha é definida pelo manipulador \texttt{endl}
	\item Este manipulador também esvazia o \emph{buffer} de saída
	\item O operador \texttt{\textless{}\textless} pode ser encadeado, permitindo uma combinação de constantes, variáveis e expressões
\end{itemize}
\begin{lstlisting}
#include <iostream>

using namespace std;

int main() {
  int x=10, y=20;
  cout << "Soma de x e y: " << x+y << endl;
  return 0;
}
\end{lstlisting}
\end{frame}

%-------------------------------------------------------
\begin{frame}[fragile]\frametitle{Erro}
\begin{itemize}
	\item \texttt{std::cerr} ou \texttt{cerr} é o fluxo de saída de erro padrão
	\item O operador \texttt{\textless{}\textless} (de inserção) também é usado para enviar informações para a saída de erro
\lstinputlisting{src/output.cpp}
	\item Compile (por exemplo: \texttt{g++ -std=c++11 output.cpp -o output}) e execute:\\
\texttt{./output \textgreater{} arquivo.txt ~ \# ou: ./output 1\textgreater ~ arquivo.txt}\\
\texttt{./output 2\textgreater{} arquivo.txt}\\
\texttt{./output \&\textgreater{} arquivo.txt}\\
\end{itemize}
\end{frame}

%-------------------------------------------------------
\begin{frame}[fragile]\frametitle{Entrada}
\begin{itemize}
	\item \texttt{std::cin} ou \texttt{cin} é o fluxo de entrada padrão
	\item O operador \texttt{\textgreater{}\textgreater} (de extração) é usado para obter informações da entrada
\end{itemize}
\begin{lstlisting}
#include <iostream>

int main() {
  int x;
  std::cout << "Lendo um inteiro: ";
  std::cin >> x;
  std::cout << "Valor lido: " << x << std::endl;
  return 0;
}
\end{lstlisting}
\end{frame}

%-------------------------------------------------------
\begin{frame}[fragile]\frametitle{Verificando a Consistência da Entrada}
\begin{lstlisting}[basicstyle=\ttfamily\scriptsize]
#include <iostream>
using namespace std;
int main() {
  int x;
  cout << "Digite um inteiro: ";
  cin >> x;
  if (cin.bad()) {
     cerr << "Houve uma falha na leitura!" << endl;
     return 1;
  }
  if (cin.fail()) {
     cerr << "Dado digitado incompativel com o dado esperado!" << endl;
     return 1;
  }
  cout << endl;
  cout << "Valor lido: " << x << endl;
  return 0;
}
\end{lstlisting}
\end{frame}

%-------------------------------------------------------
\begin{frame}[fragile]\frametitle{Entrada de \emph{Strings}}
\begin{itemize}
	\item O operador de extração ignora carateres como tabulações, espaços em branco e novas linhas
	\item Para ler \emph{strings} que englobem essas situações, usar \texttt{getline}
\end{itemize}
\begin{lstlisting}[basicstyle=\ttfamily\scriptsize]
#include <iostream>

using namespace std;

int main() {
  char frase[30];
  cout << "Digite uma frase: " << endl;
  cin.getline(frase, 30);
  cout << "Frase digitada: " << frase << endl;
  // OU
  string texto;
  cout << "Digite algum texto: " << endl;
  getline(cin, texto);
  cout << "Texto  digitado: " << texto << endl;
  return 0;
}
\end{lstlisting}
\end{frame}

%-------------------------------------------------------
\begin{frame}\frametitle{Manipuladores de Fluxos}
\begin{itemize}
	\item Em \texttt{iomanip} estão definidos diversos manipuladores, como, por exemplo:
	\begin{itemize}
		\item Para formatos de valores inteiros: \texttt{hex}, \texttt{oct}, \texttt{setbase}
		\item Para ponto flutuante: \texttt{fixed}, \texttt{setprecision}
		\item Definição de tamanho: \texttt{setw(int size)}
		\item Alinhamento: \texttt{left}, \texttt{right}
		\item Preenchimento: \texttt{setfill(char c)}
	\end{itemize}
\end{itemize}
\end{frame}

%-------------------------------------------------------
\begin{frame}[fragile]\frametitle{Manipuladores de Fluxos (Exemplo)}
\begin{lstlisting}
#include <iostream>
#include <iomanip>
using namespace std;
int main() {
  int x = 10;
  cout << hex << x << endl;
  cout << oct << x << endl;
  cout << setbase(10) << x << endl;
  cout << "Numero = " << setw(10) << right << x << endl;
  cout << "Numero = " << setw(10) << left << x << endl;
  double pi = 3.14159265;
  cout << fixed << setprecision(5);
  cout << pi << endl;
  return 0;
}
\end{lstlisting}
\end{frame}

%=======================================================
\section{Similaridades entre C e C++}

%-------------------------------------------------------
\begin{frame}\frametitle{Similaridades entre C e C++}
\begin{itemize}
	\item Todas as \textbf{estruturas de controle de fluxo} estudadas para a linguagem C podem ser utilizadas na linguagem C++
	\item Todas as \textbf{estruturas de dados} estudadas para a linguagem C podem ser utilizadas na linguagem C++
\end{itemize}
\end{frame}

%-------------------------------------------------------
\begin{frame}[fragile]\frametitle{Estruturas de Seleção: \texttt{if}}
\begin{lstlisting}
#include <iostream>

using namespace std;

int main() {
  int x, y;
  cin >> x >> y;
  if (x == y)
     cout << "x = y" << endl;
  else if (x > y)
     cout << "x > y" << endl;
  else
     cout << "x < y" << endl;
  return 0;
}
\end{lstlisting}
\end{frame}

%-------------------------------------------------------
\begin{frame}[fragile]\frametitle{Estruturas de Seleção: \texttt{switch}}
\begin{lstlisting}[basicstyle=\ttfamily\scriptsize]
#include <iostream>

using namespace std;

int main() {
  int dia;
  cin >> dia;
  switch (dia){
     case 1:  cout << "domingo"       << endl; break;
     case 2:  cout << "segunda-feira" << endl; break;
     case 3:  cout << "terca-feira"   << endl; break;
     case 4:  cout << "quarta-feira"  << endl; break;
     case 5:  cout << "quinta-feira"  << endl; break;
     case 6:  cout << "sexta-feira"   << endl; break;
     case 7:  cout << "sabado"        << endl; break;
     default: cout << "INVALIDO"      << endl;
  }
  return 0;
}
\end{lstlisting}
\end{frame}

%-------------------------------------------------------
\begin{frame}[fragile]\frametitle{Estruturas de Repetição: \texttt{while}}
\begin{lstlisting}
#include <iostream>

using namespace std;

int main() {
  int a = -1;
  while (a < 0) {
    cout << "Valor positivo: ";
    cin >> a;
  }
  cout << a << endl;
  return 0;
}
\end{lstlisting}
\end{frame}

%-------------------------------------------------------
\begin{frame}[fragile]\frametitle{Estruturas de Repetição: \texttt{do/while}}
\begin{lstlisting}
#include <iostream>

using namespace std;

int main() {
  int a;
  do {
    cout << "Valor positivo: ";
    cin >> a;
  } while(a < 0);
  cout << a << endl;
  return 0;
}
\end{lstlisting}
\end{frame}

%-------------------------------------------------------
\begin{frame}[fragile]\frametitle{Estruturas de Repetição: \texttt{for}}
\begin{lstlisting}
#include <iostream>

using namespace std;

int main() {
  int x, y;
  for (x=0, y=10; x<10; x++, y--)
      cout << "x = " << x << " y = " << y << endl;
  cout << "Valor de x e y apos o laco: " << x << ", " << y << endl;
  return 0;
}
\end{lstlisting}
\end{frame}

%-------------------------------------------------------
\begin{frame}[fragile]\frametitle{Estruturas de Dados: Vetores e Matrizes}
\begin{lstlisting}
#include <iostream>

using namespace std;

int main() {
  int i, j, v[10], m[4][4];
  for (i=0; i<10; i++)
      v[i] = i + 1;
  for (i=0; i<4; i++)
      for (j=0; j<4; j++)
          m[i][j] = i + j;
  // ...
  return 0;
}
\end{lstlisting}
\end{frame}

%-------------------------------------------------------
\begin{frame}[fragile]\frametitle{Estruturas de Dados: Registros}
\begin{lstlisting}[basicstyle=\ttfamily\scriptsize]
#include <iostream>
#include <string.h>

using namespace std;

typedef struct {
  char nome[41];
  int altura;
  double peso;
} pessoa_t;

int main() {
  pessoa_t joao;
  strcpy(joao.nome,"Joao Paulo");
  joao.altura = 181;
  joao.peso = 85.4;
  cout << joao.nome << " (" << joao.altura << "," << joao.peso << ")" << endl;
  return 0;
}
\end{lstlisting}
\end{frame}

%-------------------------------------------------------
\begin{frame}[fragile]\frametitle{Estruturas de Dados: Registros (com \texttt{string})}
\begin{lstlisting}[basicstyle=\ttfamily\scriptsize]
#include <iostream>
#include <string>

using namespace std;

typedef struct {
  string nome;
  int altura;
  double peso;
} pessoa_t;

int main() {
  pessoa_t joao;
  joao.nome = "Joao Paulo";
  joao.altura = 181;
  joao.peso = 85.4;
  cout << joao.nome << " (" << joao.altura << "," << joao.peso << ")" << endl;
  return 0;
}
\end{lstlisting}
\end{frame}

%=======================================================
\section{Exercícios}

%-------------------------------------------------------
\begin{frame}\frametitle{Exercício 1}
\begin{enumerate}
	\setItemnumber{1}
	\item Usando a linguagem C++, escreva um programa que permita armazenar o nome, a altura e a data de nascimento de até 10 pessoas. Cada pessoa deve ser representada por uma \texttt{struct} dentro de um vetor. A data de nascimento também deve ser uma \texttt{struct} que contém os campos dia, mês e ano. O programa deve iniciar perguntando se o usuário deseja inserir uma pessoa no sistema (respeitando o limite de 10 pessoas), lendo as suas informações em caso positivo. O nome, a altura e a data de nascimento de cada pessoa devem ser informados pelo teclado. Caso o usuário não queira mais inserir pessoas ou caso o limite de 10 pessoas tenha sido atingido, o programa deverá ler a data do dia de hoje e mostrar a lista das pessoas (ordenada pelo nome de forma crescente) com o seu nome, altura e idade (considerando a data fornecida como referência para o dia de hoje).
\end{enumerate}
\end{frame}

%-------------------------------------------------------
\begin{frame}[fragile]\frametitle{Exercícios 2 e 3}
\begin{enumerate}
	\setItemnumber{2}
	\item Usando a linguagem C++, escreva um programa capaz de ler os dados de uma matriz quadrada de inteiros. Ao final da leitura o programa deverá imprimir o número da linha que contém o menor dentre todos os números lidos.
		\item Usando a linguagem C++, escreva um programa que apresenta a seguinte saída, perguntando ao usuário o número máximo (no exemplo, 9). Este número deve ser sempre ímpar.
\begin{verbatim}
1 2 3 4 5 6 7 8 9
  2 3 4 5 6 7 8
    3 4 5 6 7
      4 5 6
        5
\end{verbatim}
\end{enumerate}
\end{frame}

%-------------------------------------------------------
\begin{frame}\frametitle{Exercícios 4, 5 e 6}
\begin{enumerate}
	\setItemnumber{4}
	\item Usando a linguagem C++, escreva um programa capaz de ler dois nomes de pessoas e imprimi-los em ordem alfabética. Dica: \url{http://www.cplusplus.com/reference/string/string/compare/}.
	\item Usando a linguagem C++, escreva uma função capaz de substituir todos os números negativos de uma matriz por seu módulo.
	\item Usando a linguagem C++, escreva uma rotina que remova um caracter de uma \emph{string} do tipo \texttt{char str[100]}, dada a posição do caracter.
\end{enumerate}
\end{frame}

%=======================================================
\section{Créditos}

%-------------------------------------------------------
\begin{frame}\frametitle{Créditos}
\begin{itemize}
	\item Estas lâminas contêm trechos de materiais disponibilizados pelos professores Rafael Garibotti e Matheus Trevisan.
\end{itemize}
\end{frame}

%-------------------------------------------------------
\end{document}

